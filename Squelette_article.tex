\documentclass[a4paper, 12pt]{article}


%Gestion biblio
\bibliographystyle{molecularEcology}
\usepackage{natbib}
% \usepackage[colorlinks = false, pagebackref = true]{hyperref}
% \usepackage{url}
\setcitestyle{citesep={;},aysep={}}

\usepackage{hyperref}

\usepackage[english]{babel}
\usepackage{enumitem}
\usepackage[margin=2.5cm]{geometry}


%Gestion des tables
\usepackage{array}
\usepackage{rotating}
\usepackage{afterpage}
\usepackage{multirow}
\usepackage{adjustbox}
\usepackage{booktabs}

\newcommand{\HRule}{\rule{\linewidth}{0.5mm}}

\usepackage{setspace}
\onehalfspace


% Equations
\usepackage{amsmath}
\usepackage{multicol}

% Couleurs
\usepackage{xcolor}

%Gestion grahique
\usepackage{graphicx}
\usepackage{mathspec}
\graphicspath{{./Figures/}}


% Info pour page de titre
% \author{Rachel Dubourg \\ Département de biologie \\ Université Laval}
% \title{Modélisation de systèmes ongulés-loup à l'Est et l'Ouest du Canada. \\
% \small Modification du modèle initial et observations}
% \date{Août 2022 - Session été}

\begin{document}


% \maketitle

\begin{titlepage}
	\centering
	\vspace*{1cm}
	\Huge
	\textbf{Modélisation de systèmes ongulés-loup à l'Est et l'Ouest du Canada}

	\vspace{0.5cm}
	\LARGE
	Etude de l'impact de la présence d'une proie alternative et de la productivité du milieu

	\vspace{1.5cm}

	\textbf{Rachel Dubourg}

	\vfill

	Session été \\
	Août 2022

	\vspace{0.8cm}

	\Large
	Département de Biologie\\
	Université Laval \\
	Québec, Canada \\
	2002 - 08 - 19
\end{titlepage}

\newpage
% Avant propose
% \input{AP.tex}

\newpage

% Introduction et contexte
% \input{Intro.tex}
\section{Introduction}
 \begin{itemize}
     \item Modèle et biologie
     \item Contexte spécifique : les espèces/ milieu
     \item Problématique spécifique au contexte
 \end{itemize}

\newpage
% Materiel et methodes
\section{Matériel et méthodes}
%!TEX root = ./Squelette_article.tex

\section{Explication du modèle mathématique}

We used ordinary differential equations (ODE) compartmental mechanistic and determinist model, using on , based on \cite{lamirande2022}. For every species, it includes a growth term, a density-dependent term that limits the growth of the species regarding its carrying capacity, and a mortality term, which is here only due to predation.

Here, our model relies on three trophic levels: vegetation, herbivores (primary consumers) and predator (secondary consumers). We also adapt our model to fit either the Eastern Canadian boreal forest, where only the moose and the caribou are present as prey, and the Western Canadian boreal forest, where deer represent an alternative prey that the predator can feed upon.  Les espèces présentes et leurs interactions sont représentées à la figure: \textcolor{red}{faire schema}. This model can reflect the strength of predation on caribou depeding on the density of the other present prey. 


Overall, the equations have the general form: 
\begin{equation}
    Species growth rate = growth - density-dependant limitation - mortality
\end{equation}

Two types of vegetation are available: lichen for caribou and deciduous for moose and deer. We do not consider any competition between vegetation species, but we model the relatively slow growth of lichen compared to deciduous \textcolor{red}{source}. Both species have a limited growth, constrainted by their respective carrying capacities.
\textcolor{red}{Préciser que ces capacités limites peuvent changer avec les perturbations???}

Herbivores are separated into two age class, as juveniles are significantly more vulnerable to predators, as they are less able to run from them and unable to defend themselves \textcolor{red}{source}. This distinction also aims at ... as we know caribou population growth rate is mainly influenced by reprodction rate \textcolor{red}{source}. Juvenile recruitment depends on "extra" energy that adults are able to accumulate, after providing enough energy for their own survival. Thus, the model reflects the need to have a "good" environmental quality as a prior for reproduction, as most K-strategy species do 
\textcolor{red}{source}. 


\subsection{Choix des valeurs des paramètres + citer littérature dont viennent les valeurs}
%\subsection{Model parametrisation}
Our simulations were done for two typical Canadian boreal forest sites: one located in the West (Alberta) and the other in the East (Québec). 

In the East, our model simulates the abundance of moose (\textit{Alces alces}), woodland caribou (\textit{Rangifer tarandus}), and wolf (\textit{Canis lupus}). In this system, the two ungulates are not considered to compete: caribou feed only on lichen whereas moose feed on deciduous, such as bushes or leaves. The eastern region of the Canadian boreal forest is considered to have a rather low level of disturbances, even though some populations are more exposed to them (\cite{johnson2019}).

On the contrary, West Canada has a high density of seismic lines and frequent fire cycle, with forests, replaced every 60-100 years (\cite{johnson2019, stewart2020}). As pointed out by \cite{stewart2020}, industrial disturbance density in the caribou range occurred mainly after 1980. For example, if less than 5\% of the Cold Lake caribou range was disturbed by industrial activities in 1980, this value was nearly 80\% in 2006, which stands for an increase of ~3\% of the caribou range disturbed each year.

Moreover, white-tailed deer, another ungulate species, is present and acts as an alternative prey for predators (==source==). This species also feeds on deciduous. Altogether, those conditions are rather unfavorable for caribou and most western caribou populations are in decline (\cite{stewart2020}).


\subsection{Table résumée}


\subsection{Explication des différents scenarios testés}


\section{Étude de sensibilité }
\begin{itemize}
    \item pour 1 réaction pour un paramètres,
    \item un paramètre agrégé : purement math, évite la complexité de réponse à des paramètre multi dimensionelle couplé,
    \item deux paramètres expérimentalement établis comme dominants: surface de réponse (probablment pas faisable en fait).
    \item voir Serrouya 2020 mais varaition des params de 15\% + AFC/ACP?
\end{itemize}



Log-response ratios. A follow-up GLM was employed using LRRs (log-response ratios) of exclusion treatments to investigate the interspecific variation in bird predation effects across all host plant species (Singer et al. 2012). LLRs, when used to evaluate the effects of natural enemy exclusion, provide insight into whether the interaction strength of top-down effects vary according to different environmental variables (Chaguaceda et al. 2021, Wooton 1997). 


\newpage
% Résultats
% \input{Resultats.tex}

% \section{Analyse et résultats marquants.}


% Ne pas trop s'éloigner de l'élément qui à initier ce papier: le comportement de prédation (innatendu?).

\newpage
% Discussion
% \input{Discussion.tex}

% \input{Conclusion.tex}
% 

% Annexe 
% \input{Annexe.tex}


\newpage
% Biblio
\bibliography{Biblio.bib}





\end{document}