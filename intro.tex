%!TEX root = ./Squelette_article.tex

\section{Général - Environ 1 paragraphe:}
Role preponderant des relations predateurs proies dans les syst'emes]

En particulier, comment ils permettent la stabilisation des systemes et suppotrent une + forte complexite

Trouver un moyen de parler de la competition apparente/mutualisme apparent


\section{Contexte Caribou - Environ 1 paragraphe:}

Faire le lien avec le fait que c'est l'un des facteurs majeur du déclin des caribous (DMAC; \cite{neufeld2020})

Contexte avec les perturbations anthropiques qui font que c'est exarcerbe: augmentation des fueillus et des moose etc

Parler de l'etat de conservation du caribou et des enjeux voire methodes utilisees pour l'instant , comme par exemple la reduction des mooose (voir \cite{serrouya2019}) ou la reduction de predateurs (\cite{hervieux2014}). 
Déclin dans l'ens du pays



\section{Quelles menaces - Environ 1 paragraphe:}

Bref overveiw des tentatives de modelo de ce systeme? \cite{serrouya2020} et de ce qu'ils ont permi de montrer 


Pourtant aucune ne s'est interesse a l'impact de changement de la vegetation sur les dynamiques proies predaterurs

Quelles menaces?
Zoom sur ce qu'on investigue ici en particulier (ie role de la productivite sur les forces d'interactions trophiques entre proies et predateurs)
Parler de ce qu'on sait pour le moment de ce problème \cite{chaguaceda2021}

L'hypothese princiaple est que l'agumentation des feuillus augmete le risque de predation sur le caribou...

Objectifs de notre analyse
Pour reduire les incertitudes liees a l'effet des changements globaux sur le systeme loup-orignal-caribou, notre objectif etait de predire les changements dans la pression de predation sur le caribou en fonction de la productivite du systeme. 

Knowledge gap
Parler de ce qu'on sait pour le moment de ce problème \cite{chaguaceda2021}


Hypothèses + Comment on a testé cette hypothèse
Notre premiere hypothese etait que l'augmentation l'augmentation de la qte de feuillus via l'augmentation de la productivite, entraine une augmentation de la pression de predation sur le caribou, en supportant plus d'orignaux et donc de loups dans le systeme. Nous avons teste cette hypothese en utilisant un systeme d'equations differentielle, qui permette de d'evaluer l'effet du changement de quantite de vegetation sur les dynamiques proies-predateurs du systeme. Dans notre modele, nous lions de maniere directe la productivte a l'abondance des feuillus, en accord avec les predictions (\textcolor{trouver ref}). Cependant, nous suggerons que la competition apparente generee par l'augmentation de la productivite pourrait finir par se transformer en mutualisme apparent, ou un nombre suffisament eleve de proies dans le systeme permettrait de diluer le risque de predation sur le caribou. En s'interrogeant sur les conditions qui declenchent 

qu'est-ce que tes résultats nous apprennent? Quelle est la contribution scientifique de l'étude?" 
- Peu d'évidence pour le moment de l'impact des changements de conditons du milieu sur les systèmes trophiques ayant de la compétition apparente \cite{chaguaceda2021}

- Étude (rare) d'une chaine trophique avec prédateur omnivore \cite{chaguaceda2021}

- Notre travail montre l'importance de prendre en compte les voies aternatives (ie l'ensemble des proies disponibles) lorsque le prédateur est partage pour avoir une vision d'ensemble



Qu'est-ce que dit ton étude sur le lien entre la compétition apparente et les perturbations?
- si les perturbations peuvent augmenter le risque et l'intensité de la compétition apparente, elles peuvent peut etre aussi avoir l'effet inverse et diminuer l'intensité de la compétition apparente
- peut être que la relation entre compétition apparence et les perturbations n'est pas linéaire (~ cohérent avec Frtin2017)

- les perturbations n'engendrent pas les mêmes effets en fonction de la complexité des chaines trophiques:
	- Systèmes + complexes permettent de diluer + tôt la pression de prédation 


	- Expliquer ce que sont les réponses numériques et fonctionnelles???