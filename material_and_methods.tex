%!TEX root = ./Squelette_article.tex

\section{Explication du modèle mathématique}

We used a compartmental mechanistic and determinist model, using on ordinary differential equations (ODE), based on \cite{lamirande2022}. For every species, it includes a growth term, a density-dependent term that limits the growth of the species regarding its carrying capacity, and a mortality term, which is here only due to predation.

Here, our model relies on three trophic levels: vegetation, herbivores (primary consumers) and predator (secondary consumers). We also adapt our model to fit either the Eastern Canadian boreal forest, where only the moose and the caribou are present as prey, and the Western Canadian boreal forest, where deer represent an alternative prey that the predator can feed upon.  Les espèces présentes et leurs interactions sont représentées à la figure: \textcolor{red}{faire schema}


Therefore, we have: 
\begin{equation}
    Species growth rate = growth - density-dependant limitation - mortality
\end{equation}

In our system, herbivores rely only on two types of vegetation: lichen for caribou and deciduous for moose and deer. Because it has a relatively slow growth, lichen is here modeled with a logistic growth whereas deciduous is faster to grow and is therefore modeled with exponential growth.


For each herbivore species, juvenile and adult life stages are modeled. Indeed, studies have shown that  young are significantly more vulnerable to predators, as they are less able to run from them and unable to defend themselves (==source==).

Juvenile herbivore compartments are composed likewise:

\begin{equation}
    Juvenile growth rate = Conversion rate of energy invested by parents for reproduction into newborns \times Energy not invested in basal metabolism  - Individuals becoming adults - Density-dependant limitation - Mortality
\end{equation}


Therefore, the adult’s compartment is:

\begin{equation}
    Adult growth rate = Individuals becoming adults - Density-dependant limitation - Mortality
\end{equation}


Finally, the numerical response of the predator, i.e the number of new predators generated is directly dependent on the number of prey consumed (functional response). Here, we hypothesized that the wolf had a type II functional response (\cite{holling1992}, ==source==). Thus, the wolf population can be modeled with the equation:

\begin{equation}
    Wolf growth rate = Conversion rate of energy invested by parents for reproduction into newborns \times Energy not invested in basal metabolism - Density-dependant limitation
\end{equation}

More details about the equations and their parameters can be find in the annex. Those equations were resolved using 4th-order Runge Kutta, for 2000 years projections.

\subsection{Parameter values}


\section{Mathématiques}
\begin{itemize}
    \item Modèle math
    \item Périmètre de validité et d'étude des simu.: limites de la pertinence du modèle (Analyse de calibration)
\end{itemize}

%Validité de l'étude...? Est-ce que toutes les combinaison de tous les paramètres sont toujours valides?
Pourcentage qui fit avec Messier?
Dire aussi qu<on a simule des densites qui ne sont pas observables a priori dans la vraie vie


\subsection{Présentation des équations et des paramètres dedans}


L'ensemble des équations utilisées dans ce modèle sont décrites ci-après : 

\begin{equation}
\begin{split}
\label{eq:ens_equations}
	\frac{dV}{dt} &= v_0\times V \times (1-\frac{V}{k_V}) - E_{VN} \times (N_J + N_A) \\
	\frac{dU}{dt} &= u_0 \times (1 - \frac{U}{k_U}) - E_{UM} \times (M_J + M_A) - E_{UC} \times (C_J + C_A)\\
	\frac{dW}{dt} &= w_0 \times (1 - \frac{W}{k_W}) - E_{QW} \times Q\\
	\frac{dM_J}{dt} &= \chi_M \times (e_{UM}E_{MU} - \mu_M) \times M_A - \tau_M M_J - \frac{m_0}{k_M} \times M_J(M_J + M_A) - E_{M_{J}P} \times P - E_{M_{J}Q} \times Q \\ 
	\frac{dM_{A}}{dt} &= \tau_M \times M_J - \frac{m_0}{k_M} \times M_A(M_J + M_A) - E_{M_{A}P} \times P \\
	\frac{dN_J}{dt} &= \chi_N \times (e_{VN}E_{VN} - \mu_N) \times N_A - \tau_N N_J - \frac{n_0}{k_N} \times N_J(N_J + N_A) - E_{N_{J}P} \times P - E_{N_{J}Q} \times Q \\
	\frac{dN_{A}}{dt} &= \tau_N \times N_J - \frac{n_0}{k_N} \times N_A(N_J + N_A) - E_{N_{A}P} \times P \\
	\frac{dC_J}{dt} &= \chi_C \times (e_{UC}E_{CU} - \mu_C) \times C_A - \tau_C C_J - \frac{c_0}{k_C} \times C_J(C_J + C_A) - E_{C_{J}P} \times P - E_{C_{J}Q} \times Q\\
	\frac{dC_{A}}{dt} &= \tau_C \times C_J - \frac{c_0}{k_C} \times C_A(C_J + C_A) - E_{C_{A}P} \times P \\
	\frac{dP}{dt} &= \chi_P(E_{M_{A}P} + \varepsilon_{MAJ}E_{M_{J}P} + \varepsilon_{MN}(E_{N_{A}P} + \varepsilon_{NAJ}E_{N_{J}P}) +\varepsilon_{MC}(E_{C_{A}P}+\varepsilon_{CAJ}E_{C_{J}P})-\mu_P) \times P - \frac{p_0}{k_P} \times P^2 \\
	\frac{dQ}{dt} &= \chi_Q(E_{QW} + \varepsilon_{WM}E_{M_{J}Q} + \varepsilon_{WN}E_{N_{J}Q} + \varepsilon_{WC}E_{C_{J}Q})-\mu_Q) \times Q - \frac{q_0}{k_Q} \times Q^2 \\
\end{split}
\end{equation}


\subsection{Choix des valeurs des paramètres + citer littérature dont viennent les valeurs}
%\subsection{Model parametrisation}
Our simulations were done for two typical Canadian boreal forest sites: one located in the West (Alberta) and the other in the East (Québec). 

In the East, our model simulates the abundance of moose (\textit{Alces alces}), woodland caribou (\textit{Rangifer tarandus}), and wolf (\textit{Canis lupus}). In this system, the two ungulates are not considered to compete: caribou feed only on lichen whereas moose feed on deciduous, such as bushes or leaves. The eastern region of the Canadian boreal forest is considered to have a rather low level of disturbances, even though some populations are more exposed to them (\cite{johnson2019}).

On the contrary, West Canada has a high density of seismic lines and frequent fire cycle, with forests, replaced every 60-100 years (\cite{johnson2019, stewart2020}). As pointed out by \cite{stewart2020}, industrial disturbance density in the caribou range occurred mainly after 1980. For example, if less than 5\% of the Cold Lake caribou range was disturbed by industrial activities in 1980, this value was nearly 80\% in 2006, which stands for an increase of ~3\% of the caribou range disturbed each year.

Moreover, white-tailed deer, another ungulate species, is present and acts as an alternative prey for predators (==source==). This species also feeds on deciduous. Altogether, those conditions are rather unfavorable for caribou and most western caribou populations are in decline (\cite{stewart2020}).


\subsection{Table résumée}


\subsection{Explication des différents scenarios testés}


\section{Étude de sensibilité }
\begin{itemize}
    \item pour 1 réaction pour un paramètres,
    \item un paramètre agrégé : purement math, évite la complexité de réponse à des paramètre multi dimensionelle couplé,
    \item deux paramètres expérimentalement établis comme dominants: surface de réponse (probablment pas faisable en fait).
\end{itemize}
